% --------------------------------------------------------------
% This is all preamble stuff that you don't have to worry about.
% Head down to where it says "Start here"
% --------------------------------------------------------------

\title{FooBar challenge notes}
\author{Lionel Cheng, Adrien Suau}
\date{April 2022}

\documentclass[12pt]{article}

% Math
\usepackage{amsmath,amsthm,amssymb}
\usepackage{stmaryrd}

% Physics
\usepackage{physics}
\usepackage[version=4]{mhchem}
\usepackage{chemist}
\usepackage{siunitx}

% Other
\usepackage[margin=1in]{geometry}
\usepackage[utf8]{inputenc}
\usepackage[T1]{fontenc}
\usepackage{graphicx}
\usepackage{longtable}
\usepackage{array}
\usepackage{empheq}
\usepackage{enumitem}
\usepackage{scalerel}
\usepackage{subcaption}

% For bold upgreek letters
\usepackage{upgreek}
\usepackage{stmaryrd}
\SetSymbolFont{stmry}{bold}{U}{stmry}{m}{n}
\usepackage{bm}

% Bibliography
\usepackage{natbib}

\usepackage{xcolor}
\definecolor{linkcol}{rgb}{0,0,0.4}
\definecolor{citecol}{rgb}{0.5,0,0}
\definecolor{blue_sketch}{rgb}{0,0,0.4}
\definecolor{red_sketch}{rgb}{0.5,0,0}
\usepackage{hyperref}
\hypersetup{
    colorlinks=true,
    citecolor=citecol,
    linkcolor=linkcol,
    urlcolor=linkcol,
    linktoc=all
}

% TikZ
\usepackage{tikz}
\usetikzlibrary{shapes.geometric, arrows}
\usepackage{algorithm2e}
\input{macros.tex}

% Theorems and definitions

\begin{document}
\maketitle

% --------------------------------------------------------------
%                         Start here
% --------------------------------------------------------------

\section{Challenge 4.1: Free the Bunny Workers}

\subsection{Problem}

Let us recast the problem using set notations. The inputs are:

\begin{itemize}
    \item The number of bunnies $n_r \in \intint{1}{9}$
    \item The number of required bunnies $n_r \in \intint{0}{9}$
\end{itemize}

We note $n_k$ the total number of keys which is unknown and $K = \intint{1}{n_k}$ the set of keys. The goal is to find the least lexicographical distribution of keys $D = \{d_1, \ldots, d_{n_b}\}$ such that

\begin{gather}
    \bigcup_{i \in C} d_i = K \quad \forall C \in \mc{C}(n_r, n_b) \\
    \bigcup_{i \in C} d_i \subset K \quad \forall C \in \mc{C}(n_r - 1, n_b)
\end{gather}

\noindent where $\mc{C}(n_r, n_b)$ are all the combinations of $n_r$ numbers among $n_b$ such that $\#\mc{C}(n_r, n_b) = \binom{n}{k}$.

\subsection{Reformulation of the problem}

By negating the assertion we need to find a distribution $\bar{D} = {K \setminus d_i} = \{\bar{d_1}, \ldots, \bar{d_{n_b}}\}$ such that

\begin{gather}
    \bigcap_{i \in C} d_i = \emptyset \quad \forall C \in \mc{C}(n_r, n_b) \\
    \bigcap_{i \in C} d_i \supset \emptyset \quad \forall C \in \mc{C}(n_r - 1, n_b)
\end{gather}

Since $\bigcap_{i \in C} d_i = \emptyset$ and $\# C = n_r$, it means that each key must appear \textit{at most} $n_r - 1$ times. We make the further assumptions that each key must appear \textbf{exactly} $n_r - 1$ times. That leads to the following equalities:

\begin{align}
    &\sum_{i=1}^{n_b} \mrm{Card}(\bar{d_i}) = n_k (n_r - 1)\\
    \implies & n_k = \frac{n_{kb}}{n_b - n_r + 1} n_b
\end{align}

\noindent where $n_{kb}$ corresponds to the number of keys per bunny (each bunny has the same number of keys if each key must appear $n_r - 1$ times).

\end{document}